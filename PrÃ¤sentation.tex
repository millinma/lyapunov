%\documentclass[handout]{beamer}
\documentclass[ignorenonframetext]{beamer}
\usepackage[utf8]{inputenc}
\usepackage[german, english]{babel}
\usepackage{lmodern}
\usepackage[T1]{fontenc}
\usepackage{verbatim}
\usepackage{amsmath}
\usepackage{graphicx}
\usepackage{float}
\usepackage{framed}
\usepackage{mathtools}
\usepackage[font=footnotesize]{caption}
\captionsetup{format=hang}
%\usepackage[hyperfigures =true ,linkcolor =black, urlcolor=blue, colorlinks =true, citecolor=black ,pdfauthor ={ Leonard Peter Wossnig}, ={Analytic solution and Monte Carlo simulation of the two dimensional q-state Potts model},pdfcreator ={ pdfLaTeX }]{hyperref}
%\usepackage{braket}pdftitle
%%%\usepackage[dvips]{color}
%\usepackage[pdftex]{graphicx}
%%%\usepackage{subfigure}
\usetheme{Malmoe}
\useoutertheme{infolines}
%\usepackage{mathpazo}
\parskip1.5ex

\newcommand{\changefont}[3]{\fontfamily{#1}\fontseries{#2}\fontshape{#3}\selectfont}
\definecolor{mygray}{rgb}{0.98,0.98,0.98}
\definecolor{darkgray}{rgb}{0.6,0.6,0.6}
\definecolor{mygreen}{rgb}{0.0,0.5,0.0}
\definecolor{myblue}{rgb}{0.0,0.0,0.5}

\newcommand{\bit}{\begin{itemize}}
\newcommand{\tib}{\end{itemize}}
\newcommand{\kb}{k_\mathrm{B}}
\newcommand{\wc}{\omega_\mathrm{C}}
\newcommand{\q}{q}
\newcommand{\wurzel}{q}
\newcommand{\ci}{\mathrm{i}}
\newcommand{\dif}{\mathrm{d}}


%------------------------------------------------------------------------------------
\title[]{Leaping into Lyapunov Space}
\subtitle{}
\author[Milling, Petzak]{\large{Manuel Milling, Tobias Petzak} \\
}
\institute[Universität Augsburg]{Institut für Physik der Universität Augsburg} 

\date[21.03.2016]

\titlegraphic{
\hspace{1cm}
%\includegraphics[height=2cm]{ekm}
\hspace{1cm}
%\includegraphics[height=2cm]{unia}
\hspace{1cm}
}


\beamertemplatenavigationsymbolsempty

\setbeamertemplate{headline}
{
  \leavevmode%
  \hbox{%
  \begin{beamercolorbox}[wd=1\paperwidth,ht=2.25ex,dp=1ex,left,leftskip=1em]{section in head/foot}%
    \usebeamerfont{subsection in head/foot}\hspace*{2ex}\insertsectionhead
  \end{beamercolorbox}%
  }%
  \vskip0pt%
}
\makeatletter

\setbeamertemplate{footline}
{
  \leavevmode%
  \hbox{%
  \begin{beamercolorbox}[wd=.5\paperwidth,ht=2.25ex,dp=1ex,center]{author in head/foot}%
    \usebeamerfont{author in head/foot}\insertshortauthor~~\beamer@ifempty{\insertshortinstitute}{}{}
  \end{beamercolorbox}%
  \begin{beamercolorbox}[wd=.5\paperwidth,ht=2.25ex,dp=1ex,right]{date in head/foot}%
    \usebeamerfont{date in head/foot}\insertshortdate{}\hspace*{2em}
  \end{beamercolorbox}}%
  \vskip0pt%
}
\makeatother
\begin{document}
\frame[plain]{\titlepage}

\section*{Table of Contents}
\begin{frame}
\begin{itemize}
\item Observing the phase space separation of two trajectories $\delta\Gamma(t)$, we define
\begin{equation}
\left|\delta\Gamma(t)\right|\approx \lim\limits_{\left|\Gamma_0\right| \rightarrow0} e^{\lambda t}\left|\delta\Gamma_0\right|
\end{equation}
\item $\lambda:$ Lyapunov Exponent
\item $\Gamma_0:$ initial phase space seperation (infinitesimal)
\item $\Gamma(t):$ time development of phase space separation 
\begin{equation}
\Leftrightarrow \lambda\approx \lim\limits_{\left|\Gamma_0\right| \rightarrow0} \frac{1}{t}\ln\left|\frac{\delta\Gamma(t)}{\delta\Gamma_0}\right|
\end{equation}
\end{itemize}
\end{frame}

\begin{frame}
\begin{itemize}
\item Maximal Lyapunov Exponent defined as:
\begin{equation}
\Leftrightarrow \lambda = \lim\limits_{t \rightarrow \infty} \lim\limits_{\left|\Gamma_0\right| \rightarrow0}  \frac{1}{t}\ln\left|\frac{\delta\Gamma(t)}{\delta\Gamma_0}\right|
\end{equation}
\item Considering a discrete one-dimensional system following
\begin{equation}
x_{n+1}=f(x_n)
\end{equation}
\item Lyapunov Exponent
\begin{equation}
\lambda(x_0)=\lim\limits_{N\rightarrow\infty} \frac{1}{N} \ln\left|\frac{\mathrm{d}f^N(x_0)}{\mathrm{d}x}\right|
\end{equation}
\end{itemize}
\end{frame}

\begin{frame}
\begin{itemize}
\item Using the chain rule
\begin{equation}
\frac{\dif f^N(x_0)}{\dif x}=\frac{\dif (f(f(...f(x_0))))}{\dif x}=
f'(x_N)\cdot f'(x_{N-1})\cdot ...\cdot f'(x_0)=\prod\limits_{n=0}^{N-1} f'(x_n)
\end{equation}
\item Lyapunov Exponent
\begin{equation}
\lambda(x_0)=\lim \limits_{N\rightarrow\infty}\frac{1}{N}\sum\limits_{n=0}^{N-1}\ln|f'(x_n)|
\end{equation}
\item Numerical approximation with finite $N$ 
\end{itemize}
\end{frame}

\begin{frame}
\begin{itemize}
\item implementing the population dynamics
\begin{equation}
x_{n+1} = f(x_n)=rx_n(1-x_n)
\end{equation}
\begin{equation}
f'(x_n) = r-2rx_n
\end{equation}
\begin{equation}
\lambda(x_0)= \lim \limits_{N\rightarrow\infty}\frac{1}{N}\sum\limits_{n=0}^{N-1}\ln|r-2rx_n|
\end{equation}
\end{itemize}
\end{frame}

\begin{frame}
\frametitle{Überschrift}
\begin{itemize}
\item bla
\end{itemize}
\end{frame}


\end{document}


